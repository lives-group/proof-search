\documentclass[paper=a4, fontsize=12pt]{article}

\usepackage[portuguese]{babel}
\usepackage{amsmath,amsfonts,amssymb,amsthm}
\usepackage[utf8]{inputenc}
\usepackage{color}
\usepackage{proof}

\usepackage{iftex}
\pagestyle{empty}

\ifTUTeX
  \usepackage{fontspec}
\else
  \usepackage[T1]{fontenc}
  \usepackage[utf8]{inputenc} % The default since 2018
  \DeclareUnicodeCharacter{200B}{{\hskip 0pt}}
\fi



\newtheorem{theorem}{Teorema}
\newtheorem{lemma}{Lema}
\newtheorem{corollary}{Corolário}
\theoremstyle{definition}
\newtheorem{Example}{Exemplo}
\newtheorem{Definition}{Definição}


\begin{document}

\begin{center}
  UNIVERSIDADE FEDERAL DE OURO PRETO \\
  DEPARTAMENTO DE COMPUTAÇÃO\\
\end{center}
\vspace{2cm}

\begin{center}
{\bf\Large Título:} \Large{Verificação formal de algoritmos para construção automática de provas da lógica proposicional}
\end{center}
\vspace{4cm}
\begin{flushleft}
Aluno: Felipe Peret Moraes Sasdelli\\
Orientador: Dr. Rodrigo Geraldo Ribeiro\\
\end{flushleft}
\vspace{4cm}
\noindent
Relatório Final, referente ao período 08/2020 a 07/2021, apresentado à
Universidade Federal de Ouro Preto, como parte das exigências do
programa de iniciação científica do edital EDITAL
\vspace{1cm}
\begin{center}
Ouro Preto - Minas Gerais - Brasil\\
\today
\end{center}

\clearpage

\begin{center}
{\bf\Large Resumo:}\\ \Large{Verificação formal de algoritmos para construção automática de provas da lógica proposicional}
\end{center}

\vspace{1cm}

A teoria da demonstração é o ramo do conhecimento humano devotado ao estudo de
propriedades de sistemas lógicos e de algoritmos para elaborar provas
utilizando-os [Negri01]. Além dos estudo da estrutura de provas, a
teoria da demonstração preocupa-se em estabelecer (quando possível) métodos
sistemáticos para construção de provas. Para a lógica proposicional, existem
diversos métodos para construção de demonstração e, neste projeto, estamos
interessados na verificação formal, em um assistente de provas, do cálculo de
sequentes sem a regra de contraction, conforme especificado por Dyckhoff~\cite{Dychoff92}.
Esse projeto é a continuação de nossos esforços de formalização, em assistentes de provas,
de resultados clássicos da lógica intuicionista e de sua aplicação computacional.

\begin{center}
https://github.com/lives-group/proof-search
\end{center}
\vspace{2cm}
\begin{center}
\begin{tabular}{c}
$\,$  \\
$\,$  \\
\hline
Bolsista: Felipe Peret Moraes Sasdelli\\
$\,$  \\
$\,$  \\
$\,$  \\ \hline
Orientador: Prof. Dr. Rodrigo Geraldo Ribeiro
\end{tabular}
\end{center}



\clearpage

\section{Introdução}

A lógica formal é o estudo dos princípios e critérios de inferências e
demonstrações válidas. A teoria da demonstração é o ramo do conhecimento humano
que estuda a formalização e propriedades de diferentes sistemas lógicos. Usualmente,
tais sistemas são compostos por três partes: a sintaxe (sua notação),
o seu significado (sua semântica) e regras de inferência~\cite{Negri2001}.

Dentre todos diferentes formalismos de inferência, a dedução natural é o mais
amplamente utilizado. Esse formalismo constitui a base semântica da lógica intuicionista
pois, especifica como construir provas para uma determinada proposição decompondo-a em
proposições menores. Apesar de possuir uma semântica simples, a dedução natural possui
um sério problema do ponto de vista computacional: o sistema é não determinístico, isto é,
a partir da sintaxe da fórmula a ser provada e das suposições atuais não é possível determinar,
de maneira única, qual o próximo passo de dedução a ser realizado por um algoritmo.

O cálculo de sequentes, originalmente formulado por Gentzen~\cite{Gentzen36} para demonstração de
consistência da lógica, resolve parcialmente o problema de não determinismo por reduzir o
espaço de busca das possíveis deduções para uma certa fórmula. Porém, apesar de permitir a redução
de possibilidades, o cálculo de sequentes ainda possui escolhas não determinísticas
para a construção de demonstrações. Melhorias no sentido de redução do não determinismo foram feitas
pelo trabalho de Andreolli no contexto da lógica linear com a introdução da noção de focusing, que
a aplicação de regras somente a fórmulas ditas "focused". Com isso, o não determinismo do processo
de construção automatizada de prova é eliminado e, portanto, temos que o sistema do cálculo de sequentes
com focusing constitui um algoritmo para demonstração de teoremas da lógica proposicional.

Apesar do trabalho de Andreolli apresentar uma evolução importante no desenvolvimento de algoritmos
para construção automatizada de provas, este é desnecessariamente complexo e modifica de maneira
significativa o formato de regras do cálculo de sequentes. Uma abordagem mais elegante para um
cálculo determinista é o cálculo de sequentes sem contraction de Dyckhoff~\cite{Dyckhoff92}, que modifica
apenas as regras para implicação e garante a aplicabilidade uma única regra em cada passo de dedução
(determinismo), terminação e correção em relação a versão tradicional do cálculo de sequentes.

Neste sentido, o presente projeto pretende formalizar, em um assistente de provas, o cálculo de sequentes
sem contraction e usá-lo como base para implementação de um algoritmo correto para construção de
deduções na lógica proposicional.


\section{Revisão da Literatura}


\section{Desenvolvimento}


\section{Conclusão e Trabalhos Futuros}


\bibliographystyle{plain}
\bibliography{referencias}
\end{document}
